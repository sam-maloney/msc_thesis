% Some commands used in this file
\newcommand{\package}{\emph}

\chapter{Introduction}

In the study of astrophysical entities, accretion shocks are a fairly commonly encountered physical phenomena. In particular, standing accretion shocks (SAS) are part of the theory behind core-collapse supernovae, with an instability described by Blondin et al.\ \cite{Blondin2003} being proposed as a possible mechanism for driving the explosive evolution of these phsyical systems. This instability occurs because of the spherical nature of the system, with the effects of perturbations to the symmetry of the shock being trapped in the interior subsonic region and producing feedback loops which further perturb the shock front.

In order to better understand the underlying mechanisms of this instability, Foglizzo et al.\ \cite{Foglizzo2009} and Sato et al.\ \cite{Sato2009} proposed and studied a simple toy problem, the details of which are discussed further in Section \ref{sec:toyProblem}. Using this simple set-up they showed evidence for a coupled advective-acoustic cycle between the stationary shock front and an interior decceleratiing potential step, intended to model the effects of matter settling onto the accreting object's surface.


\section{Euler Equations}
\label{sec:euler}

The time evolution of the dynamics of fluids can be described by systems of balance laws. For inviscid fluids, this system is given by the well-known Euler equations with source terms, which are given here following the notation of K\"appeli \cite{Kappeli2014} as
\begin{align} \label{eq:euler}
\frac{\partial{\rho}}{\partial{t}} &+ \nabla \cdot (\rho \mathbf{v}) = 0,\\
\frac{\partial{\rho \mathbf{v}}}{\partial{t}} &+ \nabla \cdot (\mathbf{v} \rho \mathbf{v}) + \nabla p = -\rho \nabla \phi,\\
\frac{\partial{E}}{\partial{t}} &+ \nabla \cdot \left[(E+p)\mathbf{v}\right] = -\rho \mathbf{v} \cdot \nabla \phi,
\end{align}
where $\rho$ is the mass density, and $\mathbf{v}$ is the local velocity vector. $E$ is the total energy sum of the internal and kinetic energies given as $$E=\rho e + \frac{\rho v^2}{2}.$$  An equation of state $p=p(\rho,e)$ is given to complete the relations between these primitive quanitites. The quantitiy $\phi$ on the right-hand side of the latter two equations represents an external potential field (e.g.\ gravity) which acts upon the fluid. For our purposes, this potential is assumed to be known, either as a given value, through pre-computation, or by solving for it independent from the other fluid quantities at each time step.

\subsection{Steady States}

Fun with stationary solutions.


\section{Numerics}
\label{sec:numerics}

Overview of the numerical theory

\subsection{Well-Balanced Reconstruction}

Look at the new reconstruction


\section{Toy Problem}
\label{sec:toyProblem}

Description of the Toy problem definition